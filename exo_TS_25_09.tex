\documentclass[12pt]{article}

\usepackage[utf8]{inputenc}
\usepackage[T1]{fontenc}
\usepackage[francais]{babel}
\usepackage{amsthm}
\usepackage{mathenv}
\usepackage{amsmath}
\usepackage{graphicx}
\usepackage{amssymb}
\usepackage{mathrsfs}
\usepackage{float}
\usepackage{tikz}
\usepackage{color}

\usepackage[top=0.5cm, bottom=1.5cm, left=1.5cm, right=1.5cm]{geometry}

\title{Seconde 2017/2018 : Séance 6}
\author{}
\date{}

\begin{document}

\underline{Exercice} :

\vspace{0.3cm}

On considère le patron d'un cube de côté $x$ cm, sur lequel on a placé des languettes (zones grisées) permettant l'assemblage. Ces languettes ont une largeur de 0,5 cm.

\begin{figure}[H]
\center
\begin{tikzpicture}
\fill[black!20] (-1,2) -- (-1.3,2) -- (-1.3,3) -- (-1,3);
\fill[black!20] (2,2) -- (2.3,2) -- (2.3,3) -- (2,3);
\fill[black!20] (0,0) -- (0,-0.3) -- (1,-0.3) -- (1,0);
\fill[black!20] (0,4) -- (0,4.3) -- (1,4.3) -- (1,4);

\draw[very thick] (0,0) -- (1,0) -- (1,4) -- (0,4) -- cycle;
\draw[very thick] (-1,2) -- (2,2) -- (2,3) -- (-1,3) -- cycle;
\draw[very thick] (0,1) -- (1,1);

\draw (-1,2) -- (-1.3,2) -- (-1.3,3) -- (-1,3);
\draw (2,2) -- (2.3,2) -- (2.3,3) -- (2,3);
\draw (0,0) -- (0,-0.3) -- (1,-0.3) -- (1,0);
\draw (0,4) -- (0,4.3) -- (1,4.3) -- (1,4);
\end{tikzpicture}
\end{figure}

On dispose de 504 cm$^{2}$ de matériau. On souhaite savoir pour quelles valeurs de $x$ il sera possible de construire le cube.

\vspace{0.3cm}

\begin{itemize}
\item[1-] \begin{itemize}
\item[a)] Déterminer l'aire du patron, languettes comprises.

\vspace{0.3cm}

\item[b)] Montrer que $x$ vérifie l'inéquation $\displaystyle 6x^{2} + 2x - 504 \leq 0$.

\vspace{0.3cm}

\item[c)] Justifier que l'inéquation précédente peut se réécrire $\displaystyle x^{2} + \frac{1}{3}x - 84 \leq 0$.
\end{itemize}

\vspace{0.3cm}

\item[2-] Montrer que $\displaystyle \left(x - 9 \right) \left(x + \frac{28}{3} \right) = x^{2} + \frac{1}{3}x - 84$.

\vspace{0.3cm}

\item[3-] Résoudre l'inéquation $\displaystyle x^{2} + \frac{1}{3}x - 84 \leq 0$, à l'aide d'un tableau de signes.

\vspace{0.3cm}

\item[4-] Quel est le plus grand volume que puisse avoir un cube construit de cette manière ?
\end{itemize}




\end{document}
